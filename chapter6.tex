\chapter{Discussion}

Using disease mapping problems in spatial epidemiology studies as motivation, this dissertation work extended GAM framework in order to accommodate studies that look at data over a certain period of time. In Chapter 3, we proposed time-stratified bivariate kernel smoothers and incorporated them into classic GAM framework. To test the significance of stratification, we adopted a permutation strategy, bringing in a class of PMSD tests. In Chapters 4 and 5 concentrated on disease mapping problems in longitudinal analysis. Chapter 4 filled the gap in literature and proposed a class of AMMs that model fixed effects, random effects and spatial effects simultaneously for Gaussian distributed responses. Chapter 5 further released the restriction on distribution of response and constructed GAMMs with kernel smoothers. Chapters 4 and 5 combined could be viewed as a counterpart and complementation of \citet{lin1999inference}, offering an option for researchers who aim to use LOESS or other kernel smoothers in mixed models. 

We would also like to mention that, other than a frequentist GAM framework, Bayesian disease mapping techniques, which commonly utilize Gaussian process or other types of stochastic processes for spatial effects estimation, are popular in spatial analysis as well. Bayesian methods could be more flexible if hierarchical structure is well adopted. Bayesian methods also enjoy direct inference on random effects and unified framework in inference on the whole model without approximate derivation or backfitting procedures. Nevertheless, Bayesian methods require good experience in model setup, prior distribution selection and MCMC controlling, all of which are hardly trivial to non-statisticians or even statisticians who have little expertise in geospatial analysis. In addition, when data set groups large, GAM and its extensions are generally considered more scalable. 

Along the avenue of this study, some future directions are attractive. The GAMMs in Chapter 5 used PQL procedure hence to complete the whole framework, similar work could be done for MQL procedure. Also, we noticed that in mixed models with exponential family responses, Bernoulli response in particular, the uncertainty in marginal trend parameter could not be correctly accounted for. We believe generalized estimation equations are worth investigating to solve this issue. On the other hand, this work did not consider the modeling of censored data. Therefore incorporation of the proposed methods into survival analysis framework, cox proportional hazard model in particular, should be further investigated.

%%% Local Variables: ***
%%% mode: latex ***
%%% TeX-master: "thesis.tex" ***
%%% End: ***
